\nonstopmode{}
\documentclass[a4paper]{book}
\usepackage[times,inconsolata,hyper]{Rd}
\usepackage{makeidx}
\usepackage[utf8,latin1]{inputenc}
% \usepackage{graphicx} % @USE GRAPHICX@
\makeindex{}
\begin{document}
\chapter*{}
\begin{center}
{\textbf{\huge Package `plot.counts'}}
\par\bigskip{\large \today}
\end{center}
\begin{description}
\raggedright{}
\item[Type]\AsIs{Package}
\item[Title]\AsIs{Utility functions for elementary analysis and plots of cell
counts under different conditions}
\item[Version]\AsIs{1.0}
\item[Date]\AsIs{2012-07-22}
\item[Author]\AsIs{Thomas Braschler}
\item[Maintainer]\AsIs{Thomas Braschler }\email{thomas.braschler@gmail.com}\AsIs{}
\item[Description]\AsIs{T-tests, elementary linear regression and plotting of count data}
\item[License]\AsIs{GPL-3 | file LICENCE}
\item[LazyLoad]\AsIs{yes}
\item[NeedsCompilation]\AsIs{no}
\end{description}
\Rdcontents{\R{} topics documented:}
\inputencoding{utf8}
\HeaderA{plot.counts-package}{plot.counts}{plot.counts.Rdash.package}
\aliasA{plot.counts}{plot.counts-package}{plot.counts}
\keyword{package}{plot.counts-package}
%
\begin{Description}\relax
Utility functions for comparing and elementary plotting of count data under different conditions
\end{Description}
%
\begin{Details}\relax

\Tabular{ll}{
Package: & plot.counts\\{}
Type: & Package\\{}
Version: & 1.0\\{}
Date: & 2012-07-22\\{}
License: & What license is it under?\\{}
LazyLoad: & yes\\{}
}

\end{Details}
%
\begin{Author}\relax
Thomas Braschler

Maintainer: thomas.braschler@gmail.com
\end{Author}
\inputencoding{utf8}
\HeaderA{barplot\_dots}{barplot\_dots}{barplot.Rul.dots}
\keyword{misc}{barplot\_dots}
%
\begin{Description}\relax
Bar plots for count data with errorbars and significance levels
\end{Description}
%
\begin{Usage}
\begin{verbatim}
barplot_dots(treatment,data,main=NULL,xlab=NULL,ylab=NULL,sig_codes=NULL,ylim=NULL,xpd=NA,cex.axis=NULL,cex.lab=NULL,...)
\end{verbatim}
\end{Usage}
%
\begin{Arguments}
\begin{ldescription}
\item[\code{treatment}] 
Vector describing the conditions applied. The distinct values in this vector will be the labels of the bars.

\item[\code{data}] 
Vector describing the individual results corresponding to the \code{treatment} conditions. Both \code{treatment} and \code{data} should be the raw data, not aggregated.

\item[\code{main}] 
Title of graphics (graphical parameter as described in \LinkA{par}{par})


\item[\code{xlab}] 
Label of the x-axis (graphical parameter as described in \LinkA{par}{par})

\item[\code{ylab}] 
Label of the y-axis (graphical parameter as described in \LinkA{par}{par})

\item[\code{sig\_codes}] 
Significance labels to be shown above the bars, as many entries as there are unique values in \code{treatment}

\item[\code{ylim}] 
Limits of the y-axis (graphical parameter as described in \LinkA{par}{par})


\item[\code{xpd}] 
Clipping (graphical parameter as described in \LinkA{par}{par}). Namely, supply \code{xpd=FALSE} for clipping of bars if \code{ylim} is provided with a range not starting at zero.


\item[\code{cex.axis}] 
Character size for the axis (graphical parameter as described in \LinkA{par}{par})


\item[\code{cex.lab}] 
Character size for the labels (graphical parameter as described in \LinkA{par}{par})

\item[\code{las}] 
Label orientation for axis (graphical parameter as described in \LinkA{par}{par})

\item[\code{sig.cex}] 
Character expansion for the significance label (see \LinkA{barplot\_with\_errorbars}{barplot.Rul.with.Rul.errorbars}, used internally)





\item[\code{...}] 
Additional graphical parameters, to be passed specifically to \LinkA{plot\_dot\_column}{plot.Rul.dot.Rul.column}, used internally.

\end{ldescription}
\end{Arguments}
%
\begin{Author}\relax
Thomas Braschler
\end{Author}
\inputencoding{utf8}
\HeaderA{barplot\_with\_errorbars}{barplot\_with\_errorbars}{barplot.Rul.with.Rul.errorbars}
\keyword{misc}{barplot\_with\_errorbars}
%
\begin{Description}\relax
Bar plots for count data with errorbars and significance levels
\end{Description}
%
\begin{Usage}
\begin{verbatim}
barplot_with_errorbars(height, sd_height=NULL, beside = FALSE, horiz = FALSE,group_order=NULL,sig_codes=NULL,sig.cex=1, ...)
\end{verbatim}
\end{Usage}
%
\begin{Arguments}
\begin{ldescription}
\item[\code{height}] 
As in barplot: Either a vector or matrix of values describing the bars which make up the plot. If height is a vector, the plot consists of a sequence of rectangular bars with heights given by the values in the vector. If height is a matrix and beside is FALSE then each bar of the plot corresponds to a column of height, with the values in the column giving the heights of stacked sub-bars making up the bar. If height is a matrix and beside is TRUE, then the values in each column are juxtaposed rather than stacked.

\item[\code{sd\_height}] 
Errors associated with the values, should have the same dimensions as \code{height}

\item[\code{beside}] 
As in barplot: a logical value. If FALSE, the columns of height are portrayed as stacked bars, and if TRUE the columns are portrayed as juxtaposed bars.


\item[\code{horiz}] 
As in barplot: a logical value. If FALSE, the bars are drawn vertically with the first bar to the left. If TRUE, the bars are drawn horizontally with the first at the bottom.

\item[\code{group\_order}] 
If provided, allows to reorder the columns of \code{height} and hence to change the order of the bars
\item[\code{sig\_codes}] 
Significance codes (or any other text) to be shown above the bars

\item[\code{sig.cex}] Character expansion (cex) for the significance labels. Relevant only if \code{sig\_codes} is supplied


\item[\code{...}] 
Additional graphical parameters, such as xlab, ylab, will be passed to barplot, or, like code or angle, to the \LinkA{arrows}{arrows} function used to produce the errorbars

\end{ldescription}
\end{Arguments}
%
\begin{Author}\relax
Thomas Braschler
\end{Author}
\inputencoding{utf8}
\HeaderA{barplot\_xpos}{barplot\_xpos}{barplot.Rul.xpos}
\keyword{misc}{barplot\_xpos}
%
\begin{Description}\relax
Draws a barplot with the bars at specific x-locations rather than regularly spread
\end{Description}
%
\begin{Usage}
\begin{verbatim}
barplot_xpos(height, xpos=NULL,width = 0.5,col = NULL, border = par("fg"), main=NULL, sub=NULL, xlab = NULL, ylab = NULL, xlim = NULL, ylim = NULL,  axes = TRUE, cex.axis = par("cex.axis"), plot.new=TRUE,   density=NULL, angle=45, at_x=NULL, at_y=NULL, labels_x = NULL, labels_y=NULL, ...)
                            
\end{verbatim}
\end{Usage}
%
\begin{Arguments}
\begin{ldescription}
\item[\code{height}] 
The heights of the bars, should be a vector

\item[\code{xpos}] 
The central positions of the bars, should be a vector of length identical to the \code{heights} vector

\item[\code{width}] 
Width of the bars, can be either a single number or a vector of length identical to the \code{heights} vector  

\item[\code{col}] 
Color of the bars, either single character string or vector of length identical to the \code{heights} vector

\item[\code{border}] 
Border color of the bars

\item[\code{main}] 
Main title of the plot

\item[\code{sub}] 
Subtitle of the plot

\item[\code{xlab}] 
Label for the x-axis

\item[\code{ylab}] 
Lable for the y-axis

\item[\code{xlim}] 
Limits for the x-axis, numeric vector of length 2

\item[\code{ylim}] 
Limits for the y-axis, numeric vector of length 2

\item[\code{axes}] 
Logical indicating whether axes should be drawn


\item[\code{cex.axis}] 
expansion factor for numeric axis labels (whatever that's supposed to mean, see \LinkA{barplot}{barplot}

\item[\code{plot.new}] 
Whether the bars should be added on an existing plot or whether a new plot should be drawn

\item[\code{density}] 
If provided, density of the hashing lines

\item[\code{angle}] 
If density is provided, angle of the hashing lines

\item[\code{at\_x}] If provided, tick locations for the x-axis (see \LinkA{axis}{axis})
\item[\code{at\_y}] If provided, tick locations for the y-axis (see \LinkA{axis}{axis})
\item[\code{labels\_x}] If provided, tick labels for the x-axis (see \LinkA{axis}{axis})
\item[\code{labels\_y}] If provided, tick locations for the y-axis (see \LinkA{axis}{axis})
\item[\code{...}] 
Additional graphical parameters passed on to \code{\LinkA{plot.window}{plot.window}}, \code{\LinkA{title}{title}} or \code{\LinkA{axis}{axis}}

\end{ldescription}
\end{Arguments}
%
\begin{Author}\relax
Thomas Braschler
\end{Author}
\inputencoding{utf8}
\HeaderA{change\_diagram}{change\_diagram}{change.Rul.diagram}
\keyword{misc}{change\_diagram}
%
\begin{Description}\relax
Diagram indicating the change from initial to final state by dots connected by lines
\end{Description}
%
\begin{Usage}
\begin{verbatim}
change_diagram(y_initial,y_final,x_initial=NULL,x_final=NULL,groups=NULL,plot.new=TRUE,...)
\end{verbatim}
\end{Usage}
%
\begin{Arguments}
\begin{ldescription}
\item[\code{y\_initial}] 
Initially measured values, should be a numeric vector.

\item[\code{y\_final}] 
Final measured values, should be numeric vector of length identical to \code{y\_initial}.

\item[\code{x\_initial}] 
Initial x-values. Not mandatory, but if provided, should be numeric vector of the same length as \code{y\_initial}.


\item[\code{x\_final}] 
Final x-values. Not mandatory, but if provided, should be numeric vector of the same length as \code{y\_initial}.

\item[\code{groups}] 
Group identifier. Not mandatory, but if provided, should be numeric vector, character vector or factor of the same length as \code{y\_initial}.

\item[\code{plot.new}] Start a new plot (as opposed to drawing onto some pre-existing plot





\item[\code{...}] 
Additional graphical parameters, to be passed specifically to \LinkA{plot}{plot} and \LinkA{lines}{lines} used internally. If pch, bg or col (see \LinkA{par}{par}) are provided, they can be vectors. If in addition a \code{groups} vector is provided, the values are used within the specific groups. 

\end{ldescription}
\end{Arguments}
%
\begin{Author}\relax
Thomas Braschler
\end{Author}
\inputencoding{utf8}
\HeaderA{errorbars}{errorbars}{errorbars}
\keyword{misc}{errorbars}
%
\begin{Description}\relax
Add errorbars to a plot
\end{Description}
%
\begin{Usage}
\begin{verbatim}
errorbars(x, y, sd_y = 0, angle = 90, code = 3, horiz = FALSE, ...)
\end{verbatim}
\end{Usage}
%
\begin{Arguments}
\begin{ldescription}
\item[\code{x}] 
The x-values 

\item[\code{y}] 
The nominal y-values

\item[\code{sd\_y}] 
Standard deviation of the y-values

\item[\code{angle}] 
Angle for the line ending the errorbars (if a value other than 90 is chosen, it makes arrows)

\item[\code{code}] 
Which errorbars should be drawn: \code{code=1} means only the lower errorbars (at \code{y-sd\_y}), \code{code=2} means only the upper errorbars (at \code{y+sd\_y}), \code{code=3} means both errorbars (from \code{y-sd\_y} to \code{y+sd\_y})

\item[\code{horiz}] 
Logical flag indicating whether horizontal errobars should be drawn horizontal instead of vertical. They will be centered on the same points, but extending horizontally instead of vertically

\item[\code{...}] 
Additional graphical parameters to pass down to the underlying \code{\LinkA{arrows}{arrows}} function

\end{ldescription}
\end{Arguments}
%
\begin{Author}\relax
Thomas Braschler
\end{Author}
\inputencoding{utf8}
\HeaderA{get\_associated\_values}{get\_associated\_values}{get.Rul.associated.Rul.values}
\keyword{misc}{get\_associated\_values}
%
\begin{Description}\relax
Looks up corresponding lines in a reference table
\end{Description}
%
\begin{Usage}
\begin{verbatim}
get_associated_values(descriptive_data,lookup_data,lookup_name_correspondence=NULL,FUN=NULL,lookup_value_col=NULL,...)
\end{verbatim}
\end{Usage}
%
\begin{Arguments}
\begin{ldescription}
\item[\code{descriptive\_data}] Base table indicating the measurements for which a reference value should be found. Is typically a dataframe or matrix 
\item[\code{lookup\_data}] Reference table in which the lookup should be carried out . Is typically a dataframe or matrix; usually, it will have different dimensions than the \code{descriptive\_data} table
\item[\code{lookup\_name\_correspondence}] Translation table for column correspondence in \code{descriptive\_data} and \code{lookup\_data}. Should be a matrix of two columns, the first column containing the
column names or indices that should be taken into account for \code{descriptive\_data}, the second column being the corresponding column names or indices for \code{lookup\_name\_correspondence}
\item[\code{FUN}] Aggregation function. Based on the translation table \code{lookup\_name\_correspondence}, \code{get\_associated\_values} will run row by row through \code{descriptive\_data}, trying to find matching rows
in \code{lookup\_name\_correspondence} based on the translation indicated in \code{lookup\_name\_correspondence}. For each of these lines, the element designated by \code{lookup\_value\_col} is selected, and 
\code{FUN} is applied to a vector containing all these elements. Hence, \code{FUN} should accept a vector as its input, and return a single value. By default, \code{FUN} is chosen to be \code{mean}, such that the average 
value for the matching rows in \code{lookup\_data} is calculated for every row in \code{descriptive\_data}

\item[\code{lookup\_value\_col}] Column containing the reference values of interest in \code{lookup\_data}; by default, this is assumed to be the last column of \code{lookup\_data}
\item[\code{...}] Additional arguments to be passed on to \code{FUN}
\end{ldescription}
\end{Arguments}
%
\begin{Details}\relax
The function can be thought of as an SQL join between \code{descriptive\_data} and \code{lookup\_data}, the matching condition being that the values in the columns listed in \code{lookup\_name\_correspondence} should be identical. The column to be selected in \code{lookup\_data} is given by \code{lookup\_value\_col}. As in general, there may be several rows in \code{lookup\_data} matching a given row in \code{descriptive\_data}, the values selected 
need to be aggregated into a single numerical value; this is performed by \code{FUN}   
\end{Details}
%
\begin{Value}
Of vector with length equal to the number of rows in \code{descriptive\_data}
\end{Value}
%
\begin{Author}\relax
Thomas Braschler
\end{Author}
%
\begin{Examples}
\begin{ExampleCode}
descriptive = data.frame(matching_col1=c(1,1,2,2,3,3), matching_col2=c(16,17,16,17,16,17),irrelevant_col1=c(1,1,1,1,1,1),irrelevant_col2=c(16,18,20,15,2,-1))
lookup=data.frame(match_col_1=c(1,1,1,2,2,2,3,3,3),match_col_2=c(16,17,17,16,17,17,16,17,17),value_col=c(1,2,2.5,4,3.5,3,6,5.5,NA),some_col=c(5,5,5,4,3,2,1,2,3))
lookup_name_correspondence=matrix(data=c("matching_col1","match_col_1","matching_col2","match_col_2"),ncol=2,byrow=TRUE)
lookup_value_col="value_col"
assoc_vals=get_associated_values(descriptive_data=descriptive,lookup_data=lookup,lookup_name_correspondence=lookup_name_correspondence,FUN=sum,lookup_value_col=lookup_value_col,na.rm=TRUE)
descriptive_and_associated=descriptive
descriptive_and_associated$associated = assoc_vals
cat("The base data: ")
descriptive
cat("The lookup table:")
lookup
cat("The associated values found (here, the sum of corresponding lines):")
assoc_vals
cat("The associated values in comparison with the base data:")
descriptive_and_associated
\end{ExampleCode}
\end{Examples}
\inputencoding{utf8}
\HeaderA{get\_t\_test\_matrix}{Get a T-Test matrix comparing the results under different treatments}{get.Rul.t.Rul.test.Rul.matrix}
\keyword{misc}{get\_t\_test\_matrix}
%
\begin{Description}\relax
Compares the data obtained for each value of the controlling factor, and constructs a matrix with the t-test results
\end{Description}
%
\begin{Usage}
\begin{verbatim}
get_t_test_matrix(treatment_factor, data, ...)
\end{verbatim}
\end{Usage}
%
\begin{Arguments}
\begin{ldescription}
\item[\code{treatment\_factor}] 
Factor indicating the treatment conditions

\item[\code{data}] 
Data obtained under the different conditions

\item[\code{...}] 
Additional arguments to be passed down to the underlying \code{\LinkA{t.test}{t.test}} function

\end{ldescription}
\end{Arguments}
%
\begin{Details}\relax
The full list of data should be given, so that standard deviations can be calculated. More precisely, there should be several entries where \code{treatment\_factor} has identical values.
\end{Details}
%
\begin{Value}
A matrix containing the p-values of t-tests, comparing the data for each of the possible pairwise combinations of \code{treatment\_factor}
\end{Value}
%
\begin{Author}\relax
Thomas Braschler
\end{Author}
%
\begin{Examples}
\begin{ExampleCode}
test_data=data.frame(condition=c(rep("A",5),rep("B",5),rep("C",8)),outcome=c(1,3,2,2.5,3.2,8,8.25,9,8.5,7.5,0.1,0.5,-0.5,0.2,-0.25,0,0,1))
get_t_test_matrix(test_data$condition,test_data$outcome)
\end{ExampleCode}
\end{Examples}
\inputencoding{utf8}
\HeaderA{multipleHistograms}{multipleHistograms}{multipleHistograms}
\keyword{misc}{multipleHistograms}
%
\begin{Description}\relax
Draws several histograms on the plot area. They are arranged vertically, and share a common x-axis. This a convenience function, it calls several \code{\LinkA{hist}{hist}} for each group as defined by the argument \code{levels}
\end{Description}
%
\begin{Usage}
\begin{verbatim}
multipleHistograms(x,levels, breaks ="Sturges",overhead=0.2,xlab="",category_labels=NULL,add.legend=FALSE,legend.text=NULL,
	legend.args=vector(mode="list",length=0),barcolors=NULL,cex.axis=1,FUN=NULL,...)
\end{verbatim}
\end{Usage}
%
\begin{Arguments}
\begin{ldescription}
\item[\code{x}] 
Observations for the counts, as for \code{\LinkA{hist}{hist}}. Vector of numerical values.

\item[\code{levels}] 
Assignment of the observations to the different histograms. Must be a vector of the same length as \code{x}; for each unique value in \code{levels}, a histogram will be drawn via a call to \code{\LinkA{hist}{hist}} for the counts and \code{\LinkA{barplot}{barplot}} for the actual drawing

\item[\code{breaks}] 
Algorithm or direct indication of the breaks for the histogram counts as for \LinkA{hist}{hist}

\item[\code{overhead}] 
Graphical parameter for the spacing between the figures. The larger, the more stuffed the page will appear

\item[\code{xlab}] 
Label for the x-axis as for general plotting functions (cf. \LinkA{plot}{plot})

\item[\code{category\_labels}] 
Labels to be displayed along the bottom-most histogram

\item[\code{add.legend}] 
Whether or not a legend should be added
\item[\code{legend.text}] 
Text for legend. Corresponds to the \code{legend} argument of the function \LinkA{legend}{legend} which is called for adding the legend
\item[\code{legend.args}] 
Additional arguments to be passed down to \LinkA{legend}{legend}. Must be a named list
\item[\code{barcolors}] 
Colors for the histogram. Should be a vector with as many elements as there are unique values in \code{levels}. The colors are hexadecimal values of the type "\#808AC0", with RGB coding
\item[\code{cex.axis}]  Character expansion for the axis; character expansion in R bascially means the size of some text, here it is the size of the category labels appearing under the axis)

\item[\code{FUN}] Call back function. If provided, will be called for each histogram. The idea is that in this fashion, additional things can be drawn on a histogram.\\{} This function must take three arguments: \code{FUN(level,histogram\_info,x\_vals)}. The first argument, \code{level}, indicates the level associated with the current histogram; \code{histogram\_info} gives the information concerning the current counts (corresponds to the return of \code{hist}); \code{x\_vals} gives the local x-coordinates associated with the center of the bars for the current histogram (e.g., this is the result of the local calls of \LinkA{barplot}{barplot})
\item[\code{...}] 
Additional graphical parameters to pass down to the underlying \code{\LinkA{barplot}{barplot}} function, which does the actual drawing

\end{ldescription}
\end{Arguments}
%
\begin{Author}\relax
Thomas Braschler
\end{Author}
\inputencoding{utf8}
\HeaderA{pie\_with\_errorbars}{pie\_with\_errorbars}{pie.Rul.with.Rul.errorbars}
\keyword{misc}{pie\_with\_errorbars}
%
\begin{Description}\relax
Plots a pie chart with the option of placing errorbars
\end{Description}
%
\begin{Usage}
\begin{verbatim}
	             
pie_with_errorbars(x,sd_x=NULL, labels = names(x), edges = 200, radius = 0.8, clockwise = FALSE, 
init.angle = if (clockwise) 90 else 0, density = NULL, angle = 45, 
col = NULL, border = NULL, lty = NULL, main = NULL, initiate_plot=TRUE,col_errorbar=NULL,...) 
\end{verbatim}
\end{Usage}
%
\begin{Arguments}
\begin{ldescription}
\item[\code{x}] As in \LinkA{pie}{pie}: A vector of non-negative numerical quantities. The values in x are displayed as the areas of pie slices

\item[\code{sd\_x}] Errorbars to be displayed for each slice.

\item[\code{labels}] As in \LinkA{pie}{pie}: one or more expressions or character strings giving names for the slices.
\item[\code{edges}] As in \LinkA{pie}{pie}: the circular outline of the pie is approximated by a polygon with this many edges.
\item[\code{radius}] As in \LinkA{pie}{pie}: the pie is drawn centered in a square box whose sides range from -1 to 1. If the character strings labeling the slices are long it may be necessary to use a smaller radius.
\item[\code{clockwise}] As in \LinkA{pie}{pie}: logical indicating if slices are drawn clockwise or counter clockwise (i.e., mathematically positive direction), the latter is default.
\item[\code{init.angle}] As in \LinkA{pie}{pie}: number specifying the starting angle (in degrees) for the slices. Defaults to 0 (i.e.,3 o clock) unless clockwise is true where init.angle defaults to 90 (degrees), (i.e., 12 o clock).
\item[\code{density}] As in \LinkA{pie}{pie}: the density of shading lines, in lines per inch. The default value of NULL means that no shading lines are drawn. Non-positive values of density also inhibit the drawing of shading lines.
\item[\code{angle}] As in \LinkA{pie}{pie}: the slope of shading lines, given as an angle in degrees (counter-clockwise).
\item[\code{col}] As in \LinkA{pie}{pie}: a vector of colors to be used in filling or shading the slices. If missing a set of 6 pastel colours is used, unless density is specified when par("fg") is used.
\item[\code{border}] As in \LinkA{pie}{pie}: (possibly vector) argument passed to \LinkA{polygon}{polygon} which draws each slice.
\item[\code{lty}] As in \LinkA{pie}{pie}: (possibly vector) argument passed to \LinkA{polygon}{polygon} which draws each slice.
\item[\code{main}] As in \LinkA{pie}{pie}: an overall title for the plot.
\item[\code{initiate\_plot}] If true, a new plot is drawn, otherwise the function draws onto the currently active plot
\item[\code{col\_errorbar}] As in \LinkA{pie}{pie}: Colors for the errorbars

\item[\code{...}] 
All additional arguments have the same meaning as for the underlying \LinkA{pie}{pie} function.


\end{ldescription}
\end{Arguments}
%
\begin{Author}\relax
Thomas Braschler
\end{Author}
%
\begin{Examples}
\begin{ExampleCode}
pie_with_errorbars(x=c(1,2,3),sd_x=c(0.1,0.2,0))

\end{ExampleCode}
\end{Examples}
\inputencoding{utf8}
\HeaderA{plot\_counts}{plot\_counts}{plot.Rul.counts}
\keyword{misc}{plot\_counts}
%
\begin{Description}\relax
Plots count data with errorbars, significance levels and regression model
\end{Description}
%
\begin{Usage}
\begin{verbatim}
	             
plot_counts(x, y, sd_y=NULL,sig_codes=NULL, showlinreg="NONE", plot.new = TRUE, groups=NULL, group_order = NULL, weights=NULL, category_bounds=NULL, sig.cex=1,...)
\end{verbatim}
\end{Usage}
%
\begin{Arguments}
\begin{ldescription}
\item[\code{x}] 
x values

\item[\code{y}] 
associated y values

\item[\code{sd\_y}] 
standard deviation of the y values


\item[\code{sig\_codes}] 
Codes to indicate significances levels above each data point


\item[\code{showlinreg}] 
Indicates what type of regression line should be shown. Options are "LINEAR", "SQUARE", and "NONE"


\item[\code{plot.new}] 
If true, a new plot is drawn, if false, the line is added to the existing active plot


\item[\code{groups}] 
Vector of the same length as x, y and sd\_y, indicates to which group the elements belong. Leave out if there the values are not grouped


\item[\code{group\_order}] 
Indicating the order in which the groups should be plotted. This is important for the coloring of the lines. If no \code{col} argument is provided in the optional argument, the standard colors
available via \code{palette()} will be used, otherwise the colors indicated with the "col" argument are used. The \code{col} argument should then be a vector of the same length as the \code{group\_order} argument


\item[\code{weights}] 
Explicit specification of the weights for the linear regression (useful only if a linear regression model is chosen). By default, the weights are taken to be inversely proportional to the standard deviation of the data points.


\item[\code{category\_bounds}] If supplied, the points are grouped into categories, using the \code{x} values and the \code{category\_bounds} to assign the \code{y} values. For each category, the mean \code{y} value is then displayed at the category midpoint x value, along with errorbars as calculated by \LinkA{sd\_mean}{sd.Rul.mean}


\item[\code{sig.cex}] Character expansion (cex) for the significance labels. Relevant only if \code{sig\_codes} is supplied


\item[\code{...}] 
Additional graphical parameters, such as xlab, ylab, col

\end{ldescription}
\end{Arguments}
%
\begin{Details}\relax
The function is intended to handle an x vector with ascending, unique values, and associated y data
\end{Details}
%
\begin{Author}\relax
Thomas Braschler
\end{Author}
%
\begin{Examples}
\begin{ExampleCode}
plot_counts(x=c(1,2,3),y=c(2,2,3),sd_y=c(1,1,0.5))

\end{ExampleCode}
\end{Examples}
\inputencoding{utf8}
\HeaderA{plot\_dot\_column}{plot\_dot\_column}{plot.Rul.dot.Rul.column}
\keyword{misc}{plot\_dot\_column}
%
\begin{Description}\relax
Dot-plot showing individual values, with lateral shift if the y values are too close to show all the values
\end{Description}
%
\begin{Usage}
\begin{verbatim}
	             
plot_dot_column(x,y,y_threshold_for_shifting=0.1,lateral_shift=0.06,type="p",group_order=NULL,...)
\end{verbatim}
\end{Usage}
%
\begin{Arguments}
\begin{ldescription}
\item[\code{x}] 
x values

\item[\code{y}] 
associated y values

\item[\code{y\_threshold\_for\_shifting}] 
Do lateral shift for display if two y-values with the same x-value differ by less than the threshold value


\item[\code{lateral\_shift}] 
Lateral shift to be used between the points


\item[\code{type}] 
Type argument to be passed to \LinkA{lines}{lines} used internally


\item[\code{...}] 
Additional arguments to be passed to \LinkA{lines}{lines} used internally



\end{ldescription}
\end{Arguments}
%
\begin{Details}\relax
The function is intended to handle an x vector with ascending, unique values, and associated y data
\end{Details}
%
\begin{Author}\relax
Thomas Braschler
\end{Author}
%
\begin{Examples}
\begin{ExampleCode}
plot_dot_column(x=c(1,2,3),y=c(2,2,3),sd_y=c(1,1,0.5))

\end{ExampleCode}
\end{Examples}
\inputencoding{utf8}
\HeaderA{sd\_mean}{sd\_mean}{sd.Rul.mean}
\keyword{misc}{sd\_mean}
%
\begin{Description}\relax
Calculates the standard deviation of the mean
\end{Description}
%
\begin{Usage}
\begin{verbatim}
sd_mean(x, na.rm = FALSE)
\end{verbatim}
\end{Usage}
%
\begin{Arguments}
\begin{ldescription}
\item[\code{x}] 
Data vector
\item[\code{na.rm}]  If true, NA values are removed prior to calculation 

\end{ldescription}
\end{Arguments}
%
\begin{Details}\relax
The standard deviation of the mean is the standard deviation per measurement divided by the square root of the number of measurements
\end{Details}
%
\begin{Value}
Estimated standard deviation of the mean
\end{Value}
%
\begin{Author}\relax
Thomas Braschler
\end{Author}
%
\begin{Examples}
\begin{ExampleCode}
x<-c(1,2,3,4,3.5,2.5,1)
sd(x)
sd_mean(x)

\end{ExampleCode}
\end{Examples}
\inputencoding{utf8}
\HeaderA{significance\_labels}{significance\_labels}{significance.Rul.labels}
\keyword{misc}{significance\_labels}
%
\begin{Description}\relax
Converts a numerical p-value matrix or vector into a corresponding character matrix or vector with the text labels for significance
\end{Description}
%
\begin{Usage}
\begin{verbatim}
significance_labels(x,levels=c(0.1,0.05,0.01,0.001),codes=c(".","*","**","***"))
\end{verbatim}
\end{Usage}
%
\begin{Arguments}
\begin{ldescription}
\item[\code{x}] 
P-value vector or matrix

\item[\code{levels}] 
Cut-off levels for significance

\item[\code{codes}] 
Significance codes for values below the corresponding levels

\end{ldescription}
\end{Arguments}
%
\begin{Details}\relax
\code{levels} should be sorted in decreasing order, and the \code{codes} indicate the code to show for values below or equal to the corresponding level
\end{Details}
%
\begin{Value}
A matrix or vector of dimensions identical to \code{x}, but containing the text codes for the significance level rather than the numerical p-values.
\end{Value}
%
\begin{Author}\relax
Thomas Braschler
\end{Author}
%
\begin{Examples}
\begin{ExampleCode}

significance_labels(c(0.01,0.04,0.05,0.5))
\end{ExampleCode}
\end{Examples}
\printindex{}
\end{document}
